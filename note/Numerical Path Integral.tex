\documentclass[10pt, a4paper]{article}
\usepackage[T1]{fontenc}
\usepackage[left=2cm, right=2cm, top=2cm, bottom=2cm]{geometry}
\usepackage{graphicx}
\usepackage{mathtools}
\DeclareMathOperator{\arcsinh}{arcsinh}
\usepackage{amssymb}
\usepackage{wasysym} % for \astrosun and more!
\usepackage{mathrsfs} % to use \mathscr{...}
\usepackage{xcolor} % \textcolor{red}{...} & \mathcolor{red}{...}
\usepackage{hyperref} %Automatically links \label and \ref commands; Always load last
\hypersetup{
	linktocpage,
	colorlinks=true, % false: boxed links; true: colored links
	linkcolor=red, % color of internal links
	citecolor=blue, % color of links to bibliography
	filecolor=magenta, % color of file links
	urlcolor=purple, % color of external links
} % hidelinks
%----------------------------------------------------------------------
\usepackage{fancyhdr}

\pagestyle{fancy}
\fancyhf{}  % Clear all headers and footers

% Redefine section mark to include section number and title
\renewcommand{\sectionmark}[1]{%
	\markboth{Section \thesection\ #1}{}%
}

\fancyhead[L]{\nouppercase{\leftmark}}  % Header always on the left
\fancyfoot[C]{\thepage}                 % Centered page number
%----------------------------------------------------------------------
% for XeTeX
\usepackage{xeCJK}
\setCJKmainfont{GenRyuMin JP} % 源流明体
\setCJKmonofont{GenRyuMin JP} % font used in \texttt{} (just to prevent a warning)
% for LuaTeX
%\usepackage{luatexja-fontspec}
%\setmainjfont{GenRyuMin2 JP} % 源流明体
%----------------------------------------------------------------------
% to handle a large project
\usepackage{import}
%----------------------------------------------------------------------
\usepackage{indentfirst}
\setlength{\parindent}{0em} % 首行缩进
%----------------------------------------------------------------------
% horizontal line
%\noindent\rule[0.5ex]{\linewidth}{0.5pt} % horizontal line
\usepackage{dashrule}
%\noindent\hdashrule[0.5ex]{\linewidth}{0.5pt}{1mm} % horizontal dashed line
%----------------------------------------------------------------------
\usepackage[breakable]{tcolorbox} % box with color; highlight text \colorbox{yellow}{...}
\tcbset{breakable, colback=white, colframe=green!30!black, coltext=black!60!white, fonttitle=\bfseries}

\usepackage{tabularx} % LaTeX table
\usepackage{booktabs}

\usepackage{float} % appropriate way to handle figures
% example:
%\begin{figure}[H]
%	\centering
%	\includegraphics[scale=1]{figures/file-name}
%	\caption{...}
%\end{figure}
\usepackage{subcaption}
% following David Tong's convention, one should always add a caption to figures, but not to tables.
% use \vcenter{\hbox{\includegraphics[...]{...}}} to insert figures (e.g. Feynman diagrams) in equation environment.
%----------------------------------------------------------------------
\numberwithin{equation}{section}
\allowdisplaybreaks % allow page breaks inside math environments globally

\usepackage{braket}
\usepackage{cancel}

\usepackage{leftindex}
\usepackage{tensor} % how to handle indeces: https://tex.stackexchange.com/questions/11542/left-and-right-subscript-superscript
\usepackage{isotope} % e.g. \isotoe[4][2]{He} = helium-4

\usepackage{simpler-wick} % to use Wick contraction
%\usepackage[compat=1.1.0]{tikz-feynman}

% use \, in math environment
% except for texts embeded in math environment, use \ instead
%----------------------------------------------------------------------
\usepackage{listings}
\lstset{
	frame=single,
	basicstyle=\ttfamily,
	numbers=left,
	numberstyle=\ttfamily,
}
%----------------------------------------------------------------------
\usepackage{orcidlink}
%----------------------------------------------------------------------
\title{\Huge \textbf{Numerical Path Integral}}
\author{Siyang Wan (万思扬)}
\date{June 2, 2025}
%----------------------------------------------------------------------
\begin{document}
	\maketitle
	
	\pdfbookmark{\contentsname}{toc} % add pdf bookmark to the ToC
	\tableofcontents
	
	\section{path integral in quantum mechanics}
	\begin{itemize}
		\item 考虑系统的 Hamiltonian 为
		\begin{equation}
			H = \frac{p^2}{2 m} + V(x),
		\end{equation}
		那么其 Lagrangian 为
		\begin{equation}
			L = \frac{m}{2} \dot{x}^2 - V(x),
		\end{equation}
		系统初态为 $\ket{\psi_0}$.
		
		\item 用 path integral 计算 $\psi(T, x) = \braket{x | e^{- i H T} | \psi_0}$, 有
		\begin{align}
			\braket{x | e^{- i H T} | \psi_0} =& \int Dx \, e^{i \int_0^T dt \, L} \notag \\
			=& \lim_{N \rightarrow \infty} \int dx_0 \, \psi_0(x_0) \int dx_{N + 1} \, \delta(x_{N + 1} - x) \notag \\
			& \int dx_1 \cdots dx_N \, \exp \Big( i \sum_{i = 0}^N \Delta t \Big( \frac{m}{2} \Big( \frac{x_{i + 1} - x_i}{\Delta t} \Big)^2 - V(x_i) \Big) \Big),
		\end{align}
		其中 $\Delta t = \frac{T}{N + 1}$.
		
		\item 数值计算中, 令
		\begin{equation}
			\begin{dcases}
				x_i = \Big( \frac{2 i}{M} - 1 \Big) L, \Delta x = \frac{2 L}{M}, i = 0, \cdots, M \\
				K_{i j} = \braket{x_i | e^{- i H \Delta t} | x_j} = \sqrt{\frac{m}{2 \pi i \Delta t}} \exp \Big( i \Big( \frac{m}{2} \frac{(x_i - x_j)^2}{\Delta t} - \Delta t V(x_i) \Big) \Big)
			\end{dcases},
		\end{equation}
		那么
		\begin{equation} \label{1.5}
			\braket{x | e^{- i H T} | \psi_0} = \lim_{L, M, N \rightarrow \infty} (\Delta x)^{N + 1} \sum_{j = 0}^M (K^{N + 1})_{i j} \psi_0(x_j), \quad \text{with} \quad x_i = x \ll L.
		\end{equation}
	\end{itemize}
	
	\subsection{Gaussian wave packet}
	\begin{itemize}
		\item 考虑一个自由粒子, 初态为
		\begin{equation}
			\psi_0(x) = \Big( \frac{2}{\pi} \Big)^{\frac{1}{4}} e^{- x^2 + i k_0 x}, \quad \braket{k | \psi_0} = \frac{1}{(2 \pi)^{1 / 4}} e^{- \frac{1}{4} (k - k_0)^2},
		\end{equation}
		那么, 预期结果为
		\begin{equation}
			\psi(t, x) = \Big( \frac{2}{\pi} \Big)^{\frac{1}{4}} \sqrt{\frac{m}{m + 2 i t}} \exp \Big( \frac{m}{m + 2 i t} (- x^2 + i k_0 x) - i \frac{k_0^2}{2 (m + 2 i t)} t \Big).
		\end{equation}
		
		\item 计算 \eqref{1.5} 最快 (且节省内存) 的方法是 (每一步计算都得到向量, 而不是矩阵):
		\begin{lstlisting}
psi_final = psi_0
for i in range(N+1):
	psi_final = dx * K @ psi_final
		\end{lstlisting}
		不推荐以下两种方法 (在 $T = 0.1, L = 300, M = 9000, N = 1$ 时可以得到较准确的波形):
		\begin{lstlisting}
K_power = K
for i in range(N):
	K_power = K @ K_power
psi_final = dx**(N + 1) * K_power @ psi_0
		\end{lstlisting}
		或
		\begin{lstlisting}
K_power = np.linalg.matrix_power(K, N + 1)
psi_final = dx**(N + 1) * K_power @ psi_0
		\end{lstlisting}
		
		\item 另外, 根据经验, 需要有 $\big| \Delta x \sqrt{\frac{m}{2 \pi i \Delta t}} \big|^2 \ll 1$, 可以选取 $\sim 10^{- 2}$.
		
		\item 令 $m = k_0 = 1$, 数值计算得到的结果如下图所示 (其中 $A = \int dx \, \rho$ 是数值计算得到的 normalization constant):
		
		\begin{figure}[H]
			\centering
			\begin{subfigure}{0.4\linewidth}
				\centering
				\includegraphics[scale=0.8]{figures/numerical path integral (normalized) of a free particle with initial state as a Gaussian wave packet and T=0.5 (L=400, M=12000, N=5).pdf}
				\caption{$T = 0.5$ and $A \sim 10^{11}$.}
			\end{subfigure}
			\begin{subfigure}{0.4\linewidth}
				\centering
				\includegraphics[scale=0.8]{figures/numerical path integral (normalized) of a free particle with initial state as a Gaussian wave packet and T=1 (L=400, M=12000, N=10).pdf}
				\caption{$T = 1$ and $A \sim 10^{11}$.}
			\end{subfigure}
			\caption{numerical path integral.}
		\end{figure}
	\end{itemize}
	
	\subsection{harmonic oscillator and coherent states}
	\begin{itemize}
		\item 考虑谐振子的 coherent states,
		\begin{equation}
			\begin{dcases}
				\psi^{(\alpha)}(t, x) = \Big( \frac{m \omega}{\pi} \Big)^{\frac{1}{4}} \exp \Big( - \frac{m \omega}{2} (x - \braket{x})^2 + i \braket{p} x + i \theta(t) \Big) \\
				\braket{x} = \sqrt{\frac{2}{m \omega}} \mathrm{Re}(\alpha e^{- i \omega t}), \braket{p} = \sqrt{2 m \omega} \mathrm{Im}(\alpha e^{- i \omega t}) \\
				\theta(t) = - \frac{\omega t}{2} - \mathrm{Re}(\alpha e^{- i \omega t}) \mathrm{Im}(\alpha e^{- i \omega t})
			\end{dcases},
		\end{equation}
		其中 $\alpha \in \mathbb{C}$.
		
		\item 令 $m = \omega = 1$, 考虑初始条件为
		\begin{equation}
			\alpha = |\alpha| = 1
		\end{equation}
		的情况.
		
		\item 计算中可以让
		\begin{equation}
			K_{i j} = \sqrt{\frac{m}{2 \pi i \Delta t}} \exp \Big( i \frac{m}{2} \frac{(x_i - x_j)^2}{\Delta t} \Big) \frac{e^{- i \Delta t V(x_i)} + e^{- i \Delta t V(x_j)}}{2},
		\end{equation}
		这对数值结果有微小的改进.
		
		\item 数值计算结果如下:
		
		\begin{figure}[H]
			\centering
			\includegraphics[scale=0.8]{figures/numerical path integral (normalized) of a harmonic oscillator with initial state as a coherent state and T=frac{pi}{2} (L=30, M=16000, N=15).pdf}
			\caption{numerical path integral at $T = \frac{\pi}{2}$.}
		\end{figure}
		
		注意到, coherent state 用路径积分数值计算得到的结果具有正确的归一化系数 $A = 1$.
	\end{itemize}
	
	\appendix
	\renewcommand{\sectionmark}[1]{%
		\markboth{Appendix \thesection\ #1}{}%
	}
	\section{coherent states}
	\begin{itemize}
		\item coherent states 是 annihilation operator 的本征态
		\begin{equation}
			\begin{dcases}
				a \ket{\alpha} = \alpha \ket{\alpha}, \alpha \in \mathbb{C} \\
				\ket{\alpha} = e^{\alpha a^\dag - \alpha^* a} \ket{0}
			\end{dcases}.
		\end{equation}
		
		\begin{tcolorbox}[title=proof:]
			\begin{equation}
				\begin{dcases}
					[a, (\alpha a^\dag - \alpha^* a)^n] = n \alpha (\alpha a^\dag - \alpha^* a)^{n - 1} \\
					[a^\dag, (\alpha a^\dag - \alpha^* a)^n] = n \alpha^* (\alpha a^\dag - \alpha^* a)^{n - 1}
				\end{dcases} \Longrightarrow \begin{dcases}
					[a, e^{\alpha a^\dag - \alpha^* a}] = \alpha e^{\alpha a^\dag - \alpha^* a} \\
					[a^\dag, e^{\alpha a^\dag - \alpha^* a}] = \alpha^* e^{\alpha a^\dag - \alpha^* a}
				\end{dcases},
			\end{equation}
			因此
			\begin{equation}
				\begin{dcases}
					a \ket{\alpha} = \cdots \\
					a^\dag \ket{\alpha} = e^{\alpha a^\dag - \alpha^* a} \ket{1} + \alpha^* \ket{\alpha}
				\end{dcases}.
			\end{equation}
		\end{tcolorbox}
		
		\item 并且
		\begin{equation}
			\begin{dcases}
				\braket{n | \alpha} = \\
				\braket{\alpha | \beta} = e^{- \frac{1}{2} (|\alpha|^2 + |\beta|^2 - 2 \alpha^* \beta)}
			\end{dcases}.
		\end{equation}
		
		\begin{tcolorbox}[title=calculation:]
			content...
		\end{tcolorbox}
	\end{itemize}
\end{document}